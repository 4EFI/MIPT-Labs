\documentclass[a4paper, 12pt]{article}
\usepackage{geometry}
\geometry{a4paper,
	total={170mm,257mm},left=2cm,right=2cm,
	top=1cm,bottom=2cm}

\usepackage{mathtext}
\usepackage{amsmath}
\usepackage[T2A]{fontenc}
\usepackage[utf8]{inputenc}
\usepackage[english,russian]{babel}
\usepackage{graphicx, float}
\usepackage{tabularx, colortbl}
\usepackage{caption}
\captionsetup{labelsep=period}

\newcommand{\parag}[1]{\paragraph*{#1:}}
\DeclareSymbolFont{T2Aletters}{T2A}{cmr}{m}{it}
\newcounter{Points}
\setcounter{Points}{1}
\newcommand{\point}{\arabic{Points}. \addtocounter{Points}{1}}
\newcolumntype{C}{>{\centering\arraybackslash}X}

\author{Костылев Владислав, Б01-208}
\date{\today}
\title{Отчет о выполнении лабораторной работы 1.1.4 \\ \textbf{Измерение интенсивности радиационного фона} }

\begin{document}
\maketitle
	
\begin{abstract}
	В работе используется применение методов обработки экспериментальных данных для изучения статических закономерностей при измерении интенсивности радиационного фона с использованием: счетчика Гейгера-Мюллера (СТС-6), блока питания, компьютера с интерфейсом связи со счетчиком.  
\end{abstract}
	
\section {Теоретическая справка} 
Если случайные события (регистрация частиц) однородны во времени и каждое последующее событие не зависит от того, когда и как случилось предыдущее событие, то такой процесс называется пуассоновским, а результаты -- количество отсчетов в одном опыте -- подчиняются так называемому распределению Пуассона. При больших числах отсчет это распределение стремится к нормальному. \\
Стандартная ошибка одного измерения:\\ 
\begin{equation}  
	\sigma = \sqrt{n}
\end{equation}

Формула (1) показывает, что с вероятностью 68\% измеренное число частиц n отличается от искомого среднего не более чем на $\sqrt{n}$. Результат измерений записывается так:\\
\begin{equation}  
	n_{0} = n \pm \sqrt{n}
\end{equation}

Пусть мы провели серию из N измерений, в результате которых получены числа частиц $n_{1}, n_{2}, n_{3}, \dots, n_{N}$. Эти результаты мы использовали для того, чтобы определить, как сильно значения полученные в отдельных измерениях, отличаются от истинного значения. При N измерениях среднее значение числа сосчитанных за одно измерение частиц равно:\\
\begin{equation}
	\overline{n} = \frac{1}{N}\sum_{i=1}^{N}n_{i}
\end{equation}
 
А стандартную ошибку отдельного измерения можно оценить по формуле:
\begin{equation}
	\sigma_{отд} = \sqrt{ \frac{1}{N} \sum_{i = 1}^{N} (n_{i} - \overline{n})^2 }.
\end{equation}  

В соответствии с формулой (1) следует ожидать, что эта ошибка будет близка к $\sqrt{n_{i}}$, т.е. $\sigma_{отд} \approx \sigma_{i} = \sqrt{n_{i}}$, где в качестве $n_{i}$ можно подставить любое из измеренных значений n. Ближе всего к значению   $\sigma_{отд}$, определенному по формуле (4), лежит, конечно, величина $\sqrt{ \overline{n} }$, т.е.
\begin{equation}
	\sigma_{отд} \approx \sqrt{ \overline{n} }.
\end{equation} 

Величина $\overline{n}$ из формулы (3), полученная путем усреднения результатов по серии из N опытов, конечно, тоже не вполне точно совпадает с истинным средним значением $n_{0}$ и сама является случайной величиной. Теория вероятностей показывает, что стандартная ошибка отклонения $\overline{n}$ от $n_{0}$ может быть определена по формуле:\\
\begin{equation}
	\sigma_{ \overline{n} } = \sqrt{ \frac{1}{N} \sum_{i = 1}^{N} (n_{i} - \overline{n})^2 } = \frac{\sigma_{отд}}{\sqrt{N}}.
\end{equation}

Обычно наибольший интерес представляет не абсолютная, а относительная точность измерений. Для рассмотренной серии из N измерений по 10 с относительная ошибка отдельного измерения(ожидаемое отличие любого из $n_{i}$ от $n_{0}$):\\
\begin{displaymath}
	\varepsilon_{отд} = \frac{ \sigma_{отд} }{ n_{i} } \approx \frac{ 1 }{ \sqrt{n_{i}} }.	
\end{displaymath}

Аналогичным образом определяется относительная ошибка в определении среднего по всем измерениям значения $\overline{n}$:\\
\begin{equation}
	\varepsilon_{\overline{n}} = \frac{ \sigma_{\overline{n}} }{\overline{n}} = \frac{\sigma_{отд}}{\overline{n} \sqrt{N}}\approx \frac{ 1 }{ \sqrt{\overline{n} N} }.
\end{equation}

\section {Используемое оборудование}
Космические лучи обнаруживают с помощью ионизации, которую они производят, используя счетчик Гейгера-Мюллера. Счетчик представляет собой наполненный газом сосуд с двумя электродами. Частицы космических лучей ионизируют газ, выбивают электроны из стенок сосуда. Те, сталкиваясь с молекулами газа, выбивают из них электроны. Таким образом, получается лавина электронов, следовательно, через счетчик резко увеличивается ток. 

Погрешность измерения потока частиц с помощью счетчика Гейгера-Мюллера мала по сравнению с изменениями самого потока, то есть его флуктуациями.

\section {Методика измерений} 
\begin{enumerate}
	\item Ознакомление с устройством установки. 
	\item Включение питания компьютера и установки. После загрузки компьютера -- запуск программы \textbf{STAT} и таким образом начинается проведение основного эксперимента.  
	\item По окончании основного эксперимента сохранение полученных данных в таблицу для последующей обработки. 
	\item Для каждого числа импульсов происходит подсчет числа случаев и доли случаев.  
	\item Объединения соседние ячейки таблицы 1, получаем новые данные, а именно, число срабатываний счетчика за 40 сек.  
	\item По полученным данным происходит построение гистограмм и графиков распределения Гаусса. 
\end{enumerate}

\section {Результаты измерений и обработка данных}
По завершении основного эксперимента мы получаем следующие данные:

\begin{table}[H]
	\centering
	\begin{tabular}{|r|c|c|c|c|c|c|c|c|c|}
		\hline
		Число импульсов & 3 & 4 & 5 & 6 & 7 & 8 & 9 & 10 & 11 \\ \hline
		
		Число случаев & 3 & 8 & 17 & 26 & 35 & 34 & 42 & 56 & 43 \\ \hline  
		
		Доля случаев & 0.0075 & 0.02 & 0.0425 & 0.065 & 0.0875 & 0.085 & 0.105 & 0.14 & 0.1075 \\ \hline 	
		
		Число импульсов & 12 & 13 & 14 & 15 & 16 & 17 & 18 & 19 & 20 \\ \hline
		
		Число случаев & 39 & 31 & 18 & 18 & 5 & 12 & 5 & 4 & 4 \\ \hline  
		
		Доля случаев & 0.0075 & 0.02 & 0.0425 & 0.065 & 0.0875 & 0.085 & 0.105 & 0.14 & 0.1075 \\ \hline 	
	\end{tabular}
	\caption{Данные для гистограммы, t = 10 сек}	
\end{table}

По данным построим гистограмму, а также распределение Гаусса, пользуясь формулой:\\
\begin{equation}
	f(x) = \frac{1}{\sigma \sqrt{2\pi}} \cdot e^{ \frac{ -(x - \overline{n})^2 } { 2\sigma^{2} } }
\end{equation} 

\begin{table}[H]
	\centering
	\begin{tabular}{|r|c|c|c|c|c|c|c|c|c|c|}
		\hline
		№ Опыта: & 1 & 2 & 3 & 4 & 5 & 6 & 7 & 8 & 9 & 10     \\ \hline
		
		0   & 21 & 15 & 22 & 16 & 16 & 25 & 15 & 21 & 25 & 23 \\ \hline
		10  & 18 & 27 & 14 & 18 & 18 & 21 & 24 & 20 & 24 & 13 \\ \hline
		20  & 20 & 17 & 21 & 18 & 18 & 15 & 15 & 18 & 13 & 21 \\ \hline
		30  & 17 & 16 & 22 & 29 & 19 & 22 & 23 & 25 & 20 & 11 \\ \hline
		40  & 29 & 22 & 18 & 25 & 17 & 22 & 25 & 24 & 19 & 26 \\ \hline
		50  & 27 & 31 & 25 & 15 & 17 & 26 & 24 & 25 & 20 & 22 \\ \hline
		60  & 23 & 23 & 20 & 19 & 20 & 17 & 15 & 31 & 24 & 17 \\ \hline
		70  & 28 & 19 & 19 & 18 & 26 & 31 & 22 & 26 & 23 & 29 \\ \hline
		80  & 10 & 20 & 32 & 14 & 14 & 18 & 31 & 25 & 24 & 22 \\ \hline
		90  & 22 & 20 & 21 & 13 & 23 & 30 & 13 & 13 & 17 & 21 \\ \hline
		100 & 18 & 25 & 22 & 15 & 17 & 18 & 22 & 27 & 20 & 20 \\ \hline
		110 & 20 & 19 & 18 & 28 & 14 & 12 & 17 & 18 & 19 & 25 \\ \hline
		120 & 25 & 20 & 21 & 19 & 29 & 22 & 25 & 14 & 28 & 16 \\ \hline
		130 & 13 & 19 & 14 & 13 & 17 & 19 & 22 & 19 & 22 & 20 \\ \hline
		140 & 29 & 25 & 18 & 23 & 23 & 16 & 29 & 19 & 20 & 21 \\ \hline
		150 & 25 & 19 & 16 & 26 & 15 & 14 & 24 & 21 & 25 & 26 \\ \hline
		160 & 24 & 25 & 19 & 19 & 20 & 15 & 24 & 23 & 12 & 18 \\ \hline
		170 & 23 & 22 & 17 & 24 & 24 & 34 & 21 & 19 & 18 & 18 \\ \hline
		180 & 18 & 16 & 29 & 10 & 22 & 28 & 21 & 16 & 16 & 18 \\ \hline
		190 & 19 & 24 & 15 & 27 & 17 & 14 & 28 & 24 & 16 & 23 \\ \hline
	\end{tabular}
	\caption{Число срабатываний счетчика за 20 сек}
\end{table}

Обработав данные, используя язык программирования  Python, мы получаем соответствующее число случаев и долю случаев для определенного числа импульсов.  

\begin{table}[H]
	\centering
	\begin{tabular}{|r|c|c|c|c|c|c|c|c|}
		\hline
		Число импульсов & 10 & 11 & 12 & 13 & 14 & 15 & 16 & 17 \\ \hline
		
		Число случаев & 2 & 1 & 2 & 7 & 8 & 10 & 10 & 12 \\ \hline  
		
		Доля случаев & 0.01 & 0.005 & 0.01 & 0.035 & 0.04 & 0.05 & 0.05 & 0.06 \\ \hline 	
		
		Число импульсов & 18 & 19 & 20 & 21 & 22 & 23 & 24 & 25 \\ \hline
		
		Число случаев & 19 & 17 & 15 & 12 & 16 & 11 & 13 & 16 \\ \hline  
		
		Доля случаев & 0.095 & 0.085 & 0.075 & 0.06 & 0.08 & 0.055 & 0.065 & 0.08 \\ \hline 	
		
		Число импульсов & 26 & 27 & 28 & 29 & 30 & 31 & 32 & 34 \\ \hline
		
		Число случаев & 6 & 4 & 5 & 7 & 1 & 4 & 1 & 1 \\ \hline  
		
		Доля случаев & 0.03 & 0.02 & 0.025 & 0.035 & 0.005 & 0.02 & 0.005 & 0.005 \\ \hline 	
	\end{tabular}
	\caption{Данные для гистограммы, t = 20 сек}
\end{table}

Далее взял среднее арифметическое между каждыми двумя соседними ячейками Таблицы 2, мы получаем новую таблицу данных: 

\begin{table}[H]
	\centering
	\begin{tabular}{|r|c|c|c|c|c|c|c|c|c|c|}
		\hline
		№ Опыта: & 1 & 2 & 3 & 4 & 5 & 6 & 7 & 8 & 9 & 10     \\ \hline
			
		0   & 18 & 19 & 20 & 18 & 24 & 22 & 16 & 19 & 22 & 18 \\ \hline
		10  & 18 & 19 & 16 & 16 & 17 & 16 & 25 & 20 & 24 & 15 \\ \hline
		20  & 25 & 21 & 19 & 24 & 22 & 29 & 20 & 21 & 24 & 21 \\ \hline
		30  & 23 & 19 & 18 & 23 & 20 & 23 & 18 & 28 & 24 & 26 \\ \hline
		40  & 15 & 23 & 16 & 28 & 23 & 21 & 17 & 26 & 13 & 19 \\ \hline
		50  & 21 & 18 & 17 & 24 & 20 & 19 & 23 & 13 & 17 & 22 \\ \hline
		60  & 22 & 20 & 25 & 19 & 22 & 16 & 13 & 18 & 20 & 21 \\ \hline
		70  & 27 & 20 & 19 & 24 & 20 & 22 & 21 & 14 & 22 & 25 \\ \hline
		80  & 24 & 19 & 17 & 23 & 15 & 22 & 20 & 29 & 20 & 18 \\ \hline
		90  & 17 & 19 & 25 & 18 & 17 & 21 & 21 & 15 & 26 & 19 \\ \hline
		\end{tabular}
	\caption{Число срабатываний счетчика за 40 сек}
\end{table}

Проделав те же действия программно, что и с Таблицей 2, получаем следующее:

\begin{table}[H]
	\centering
	\begin{tabular}{|r|c|c|c|c|c|c|c|c|c|}
		\hline
		Число импульсов & 13 & 14 & 15 & 16 & 17 & 18 & 19 & 20 & 21 \\ \hline
		
		Число случаев & 3 & 1 & 4 & 6 & 7 & 10 & 12 & 11 & 9  \\ \hline  
		
		Доля случаев & 0.03 & 0.01 & 0.04 & 0.06 & 0.07 & 0.1 & 0.12 & 0.11 & 0.09 \\ \hline 	
	
		Число импульсов & 22 & 23 & 24 & 25 & 26 & 27 & 28 & 29 & \\ \hline
		
		Число случаев & 9 & 7 & 8 & 5 & 3 & 1 & 2 & 2 & \\ \hline  
		
		Доля случаев & 0.09 & 0.07 & 0.08 & 0.05 & 0.03 & 0.01 & 0.02 & 0.02 &  \\ \hline 	
	\end{tabular}
	\caption{Данные для гистограммы, t = 20 сек}
\end{table} 

\section {Обсуждение результатов}


\section {Заключение}
В ходе лабораторной работы были получены случайно изменяющиеся со временем данные об интенсивности потока космических частиц, а также были применены методы обработки данных для изучения статистических закономерностей при измерении интенсивности радиационного фона.

\end{document}
